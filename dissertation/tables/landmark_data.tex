% latex table generated in R 3.6.2 by xtable 1.8-4 package
% Mon Nov 23 17:25:41 2020
\begin{longtable}{rrrr}
  \hline
 & Males & Females & Total \\ 
  \hline
\textit{Gorilla gorilla} &  48 &  36 &  84 \\ 
  \textit{Homo neanderthalensis} &   5 &   0 &   5 \\ 
  \textit{Homo sapiens} & 118 & 112 & 230 \\ 
  \textit{Macaca fascicularis} &  30 &  21 &  51 \\ 
  \textit{Macaca hecki} &  10 &   7 &  17 \\ 
  \textit{Macaca mulatta} &  11 &  12 &  23 \\ 
  \textit{Macaca sylvanus} &  10 &  10 &  20 \\ 
  \textit{Macaca tonkeana} &   9 &  11 &  20 \\ 
  \textit{Mandrillus leucophaeus} &  22 &  15 &  37 \\ 
  \textit{Mandrillus sphinx} &  19 &  12 &  31 \\ 
  \textit{Pan paniscus} &  21 &  28 &  49 \\ 
  \textit{Pan troglodytes} &  55 &  74 & 129 \\ 
  \textit{Papio hamadryas} & 264 & 136 & 400 \\ 
   \hline
\hline
\caption[Composition of Landmark Data Used in Empirical Analysis]{A table detailing the composition of the landmark dataset used in 
                                            our empirical analysis. Elements of the table represent numbers of individuals in each 
                                            row species corresponding to each 
                                            estimated column sex. Further details regarding landmarks used can be found in \citep{harvatiNeanderthalTaxonomyReconsidered2004}.} 
\label{tab:landmarkDataComposition}
\end{longtable}
