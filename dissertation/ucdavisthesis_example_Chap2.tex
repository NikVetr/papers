\chapter{The Second Chapter}

\section{Overall Appearance}
%
You are responsible for the appearance of your manuscript in PDF. It will appear and may be downloaded exactly as you submit it.

\subsection{Tables, Graphs and Captions}
%
Charts and tables may be placed horizontally or vertically, but in either case must fit within the required margins. It may be necessary to use a reducing copier in order to achieve this. If necessary, wide tables, charts, and figures can be placed sideways. Figures may be embedded in the text or take up a full page. Each figure or table must be numbered consecutively (do not renumber each chapter unless you include chapter numbers, e.g., Fig. 1.1, Fig. 2.1, etc.) and should have a caption.

NOTE: If your figures or charts are placed horizontally on the page (i.e. in “Landscape” orientation), your page number must still appear in the same place as all other page numbers (centered at the bottom of the page in “Portrait” orientation). Pagination must be consistent throughout the document.


\subsection{Photographs, Illustrations and Maps}
%
Plates, figures, illustrations, maps and photographic reproductions must be clear and distinct. Pagination must be consistent.

\subsection{Oversized Material}
%
Consult the ETD website guidelines for uploading supplemental files with your manuscript.

\subsection{Using Published Material}
%
If approved by the thesis or dissertation committee, reports of research undertaken during graduate study at UC Davis which have been published may be accepted in printed form as all or part of the master's thesis or doctoral dissertation. If you are not the sole or first author of the published material submitted, the use of co-authored materials must be approved by the department or graduate group concerned.
The pages of the published material must meet the same formatting guidelines. Each chapter may have an abstract of its own. There must be a general abstract covering the entire dissertation.

\subsection{Copyrighted Material Use}
%
Since the submission of your thesis or dissertation to the University Library and/or its being made available by PQIL may constitute a form of publication, you may have to obtain permission to use (or quote) copyrighted material, such as that in most journal articles or books. It is the author (i.e. you) who is responsible in the matter of copyrighted materials. The agreement, which you submit to PQIL, specifically absolves them of any such responsibility.

If you quote extensively from a particular author, especially in fields such as fiction, drama, criticism, or poetry, or if copyrighted maps, charts, statistical tables, or similar materials have been reproduced, you must write the copyright owner(s), describe the use which you are making of the materials, and request permission to use it in the dissertation or thesis.

For your protection, a statement listing such materials should be included in the acknowledgements of the dissertation or thesis. The statement should inform the reader (1) that permission has been granted for their use, and (2) the source of the permission.

\subsection{Dates of Filing}
%
Check the calendar for deadlines for filing the master's thesis or the doctoral dissertation with the committees in charge and with Graduate Studies. Deadlines are also announced each year in the Class Schedule and Registration Guide and the General Catalog. The deadline for filing with your committee is a recommended deadline to allow time for making revisions. The deadline for filing with Graduate Studies is firm.

It is important to bring all documents, forms and supplies with you when you file your thesis or dissertation. Please review the checklist for master's or doctoral students prior to your appointment.

\subsection{Title Page}
%
Graduate Studies does not supply the title page. You must prepare your title page in accordance with the sample. The title page is to be signed by all members of your committee when they have approved the thesis or dissertation. Only the original title page will be accepted with the thesis or dissertation.

\subsection{Dissertation Abstracts}
%
Master’s theses and doctoral dissertations are required to include an abstract. If your abstract appears in the introductory pages of your thesis/dissertation manuscript, it must follow the same format as the rest of your thesis/dissertation (1 inch margin on all sides, double-spaced, consecutive page numbering, etc.).

A separate abstract is submitted to ProQuest Information and Learning (PQIL) during the electronic submission process and must be formatted following the guidelines on the ETD website. It is important to write an abstract that gives a clear description of the content and major divisions of the thesis/dissertation, since PQIL will publish the abstract exactly as submitted.

Students completing their requirements under doctoral Plan A should provide copies of the abstract for use by the dissertation committee during the examination.

\subsection{Diploma}
%
When you file your thesis or dissertation, you will receive a Letter of Certification that states you have completed all the requirements for your degree and which will provide the official conferral date of your degree. This certificate may be given to your employer for proof of degree until the Registrar's Office issues an official transcript or diploma. You must complete a form to request your transcript or diploma. Official transcripts normally are available two months after the official degree conferral date, diplomas normally are available four months after this date.

\subsection{Copyright and Publication}
%
The copyright law of the United States is quite complex. The information contained in this section is only a general guide – more detailed information must be obtained from other sources.

Whether or not you copyright your thesis or dissertation, you retain the right to publish all or any part of it by any means at any time, except for reproduction from a negative microfilm as described in the agreement with PQIL. Should you decide to copyright your thesis or dissertation, you must include a separate unnumbered copyright page after the title page. By adding this copyright notice, which should be included in all copies you distribute, you have copyrighted your thesis or dissertation. At this point you have several options:

You may have the copyright registered for you by PQIL. Along with the UMI Doctoral Dissertation Agreement, you will need to submit a fee to cover the copyright cost.

You may register the copyright yourself by submitting to the Registrar of Copyrights the appropriate application form, a filing fee and one or two copies of the work. In order to have full protection against infringement, this should be done as soon as possible. Information and forms can be obtained from the Registrar of Copyrights, Library of Congress, Washington D.C. 20559.

You may choose to copyright your thesis or dissertation by adding the copyright notice, submitting a copy to the Registrar of Copyrights, but not registering it. (Federal copyright law requires that copies of all works published with notices of copyright be deposited with the Library of Congress, even if the copyright is not registered). However, to protect your rights in a copyright dispute and in order to be compensated for damages caused by infringement, your copyright must be registered.

\subsection{Filing Fee}
%
The Filing Fee was established expressly to assist those students who have completed all requirements for degrees except filing theses or dissertations and/or taking final examinations (master's comprehensive exams or doctoral final examinations) and are no longer using University facilities. The Filing Fee is a reduced fee paid in lieu of registration fees. It is assessed only once and must be paid to the Cashier's Office prior to submission of the form to Graduate Studies. Filing Fee status restrictions (more restrictions are noted on the application instruction sheet):
%
\begin{itemize}
  \item You may not be using University facilities;
  \item You cannot be using faculty time other than the time involved in the final reading of the thesis or dissertation or in holding final examinations;
  \item You are not eligible to hold any academic appointment title for more than 1 quarter;
  \item You cannot hold a fellowship or receive financial aid.
\end{itemize}

If you are eligible to use the Filing Fee procedure, you should complete a Filing Fee application, obtain the signature of the Graduate Adviser and your Committee chair, and return the application to Graduate Studies before you stop registering. The Filing Fee must be paid prior to submitting the application to Graduate Studies.

Original (initial) filing fee deadlines adhere to registration deadlines. For example, if you were approved for one quarter of filing fee, you would be eligible to submit your dissertation or thesis up to the last day of late registration for the following quarter. If you do not submit by that date your filing fee status will expire and you would need to secure an extension from your program and from Graduate Studies. Filing fee extensions end the date noted on the petition. Make sure your filing fee is current before you submit your dissertation or thesis.
